
\begin{frame}{Model Selection: Testing Random Intercept}
To compare whether these groups have different base rates of heart disease, we test the random intercept of these four models using AIC for model selection.
    \input{tables/aic}
\end{frame}

\begin{frame}{Model Selection: Testing Random Intercept}
\begin{itemize}
    \item The four null models are preferred as they have lower AIC's, although not by much which again could be because of the small amount of data used.
    \item All the random intercepts are not statistically significant.
    \item There is little evidence that suggests the populations being compared vary in base rate odds of having heart disease.
\end{itemize}
 
\end{frame}

\begin{frame}{Model Selection: Testing Random Slope}
Similarly, we test whether individuals between the four models differ in their odds of having OA. We test these random slopes using $\chi^2$ model selection.
\end{frame}

\begin{frame}{Model Selection: Testing Random Slope}
    \input{tables/anovas}
\end{frame}
\begin{frame}{Model Selection: Testing Random Slope}
\begin{itemize}
    \item None of the random slopes are statistically significant.
    \item This suggests that the four populations being compared to do not differ in how OA effects their odds of having heart disease. 
\end{itemize}
\end{frame}
