
\begin{frame}{Model Selection: Testing Random Intercept}
To compare whether these groups have different average responses, we test the random intercept of these four models using AIC for model selection.
    
\begin{table}[ht]
	   \caption{Testing Random Intercept} 
	\label{aic}
\centering
\begin{tabular}{rrr}
  \hline
 & df & AIC \\ 
  \hline
Economic Cost Povince null & 22.00 & 47151.83 \\ 
Economic Cost Povince full & 23.00 & 46376.05 \\ 
Economic Cost Time null & 22.00 & 47151.83 \\ 
Economic Cost Time full & 23.00 & 46357.20 \\ 
Human Cost Province null & 22.00 & -1447.57 \\ 
Human Cost Province full & 23.00 & -1307.21 \\ 
Human Cost Time null & 22.00 & -1447.57 \\  
Human Cost Time full & 23.00 & -1394.44 \\ 
   \hline
\end{tabular}
\end{table}

\end{frame}

\begin{frame}{Model Selection: Testing Random Intercept}
\begin{itemize}
	\item \textbf{Linear Model with Random Intercept for Themes}: The AIC for the mixed model with random intercept is lower than the null, so it is preferred. This suggest that there exists differences in the average popularity score across themes.
	\item \textbf{Linear Model with Random Intercept for Time}: The AIC for the mixed model with random intercept is lower than the null, so it is preferred. This suggest that there exists differences in the average popularity score across time.
	\item \textbf{Poisson Model with Random Intercept for Themes}: The AIC for the mixed model with random intercept is lower than the null, so it is preferred. This suggest that there exists differences in the average avg. views per day across themes.
	\item \textbf{Poisson Model with Random Intercept for Time}: The AIC for the mixed model with random intercept is lower than the null, so it is preferred. This suggest that there exists differences in the average avg. views per day across time.
\end{itemize}
 
\end{frame}

\begin{frame}{Model Selection: Testing Random Slope}
	Similarly, we test whether individuals between the four models differ in their odds of having OA. We test these random slopes using $\chi^2$ model selection.
    % latex table generated in R 3.4.4 by xtable 1.8-4 package
% Mon Mar  9 21:33:01 2020
\begin{table}[ht]
	   \caption{Testing Random Slopes} 
	\label{anova}
\centering
\begin{tabular}{lrrrrr}
  \hline
 & Df & Chisq & Chi Df & Pr($>$Chisq) \\ 
  \hline
Economic Cost Povince simple & 23&  &  &  \\ 
Economic Cost Povince full & 253 & 0.00 & 230 & 1.0000 \\ 
Economic Cost Time simple & 23 &  &  &  \\ 
Economic Cost Time full & 253 & 935.65 & 230 & 0.0000 \\ 
Human Cost Province simple & 23 &  &  &  \\ 
Human Cost Province full & 253 & 47.41 & 230 & 1.0000 \\ 
Human Cost Time simple & 23 &  &  &  \\ 
Human Cost Time full & 253 & 1719.02 & 230 & 0.0000 \\ 
   \hline

\end{tabular}
\end{table}

\end{frame}
\begin{frame}{Model Selection: Testing Random Slope}
\begin{itemize}
	\item \textbf{Linear Model with Random Slope for Themes}: The p-value is very small; the preferred model includes the random intercept. This suggest that there exists differences in how the predictors effect popularity score across themes.
	\item \textbf{Linear Model with Random Slope for Time}: The p-value is very small; the preferred model includes the random intercept. This suggest that there exists differences in how the predictors effect popularity score across time.
	\item \textbf{Poisson Model with Random Slope for Themes}: The p-value is very small; the preferred model includes the random intercept. This suggest that there exists differences in how the predictors effect avg. views per day across themes.
	\item \textbf{Poisson Model with Random Slope for Time}: The p-value is very small; the preferred model includes the random intercept. This suggest that there exists differences in how the predictors effect avg. views per day across time.
\end{itemize}
\end{frame}

\begin{frame}{Model Selection}
	\begin{itemize}
		\item The best models include both random intercepts and slopes for both Poisson and Linear models.
		\item  Which suggests that a lot of the variation in the data can be explained by timing of when a video is published and its theme rather than the actual form/content.
		\item  The main predictors in these models are the talks duration and age (in terms of how recent since it has been published). 
		\item Which suggests that the best way to create a popular video is by having a long video and re-uploading it to escape the drop in popularity from aging.
	\end{itemize}
\end{frame}