\section{Response and Predictors}
\subsection{Responses}
In this analysis, we used two measures of popularity; the most intuitive being a form of views count, and a composite response.
\subsubsection{Average View Per Day}
The most intuitive measurement of popularity is number of views. However, the flaw of this is that it does not account for age of a video, that is an older video will have more views than a newer video just by the nature of being around longer and able to garnish more views. Indeed, this raw view measurement wold suggest that a day old video with 1 million views is just as popular as a 5 year old video with 1 million views. To account for this, we divide the number of views of a video by the number of days since it has been published. This allows videos to be equal across length since publication.
\subsubsection{Popularity}


\subsection{Predictors}
\subsubsection{Exposure/Risk Factor: Osteoarthritis}


\subsection{Covariates \& Confounders}

