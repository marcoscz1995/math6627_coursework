\section{Response and Predictors}
\subsection{Response: Heart Disease}
 The response variable is binary; whether a survey participant reported having some form of cardiovascular disease or not. Heart disease includes having had a heart attack, angina, stroke or congestive heart failure. This was determined in the CCHS by the following
question “Now I’d like to ask about certain chronic
health conditions that have lasted 6 months or more
and have been diagnosed by a health professional. Do
you have heart disease?” In CCHS cycles 1.1, 2.1 and 3.1
persons reporting heart disease were asked specific questions about MI, angina, CHF and stroke: “Have you ever
had a heart attack (damage to the heart muscle)?”, “Do
you currently have angina (chest pain, chest tightness)?”,
and “Do you currently have CHF (inadequate heartbeat,
fluid build-up in the lungs or legs)?”. In addition, CCHS
includes the question of “Do you suffer from the effects
of a stroke?” These specific questions provide the data to
perform disease-specific analyses using each of these
four health conditions as outcomes using CCHS cycles
1.1, 2.1 and 3.1. \cite{rahman2013relationship}


\subsection{Predictors}
\subsubsection{Exposure/Risk Factor: Osteoarthritis}
The main independent variable in this study is OA. Table \ref{oa_var} describes OA in the data. In the CCHS
data, OA is self reported. In the CCHS
data, OA was assessed by asking two questions. The first
question was “Now I’d like to ask about certain chronic
health conditions that have lasted 6 months or more
and have been diagnosed by a health professional. Do
you have arthritis or rheumatism?” The second question
was “What type of arthritis?” and the four response
options were Rheumatoid arthritis, OA, other, and do
not know”. Since the second question is restricted to
only arthritis patients, over-reporting of OA is minimised. \cite{rahman2013relationship}.
\input{tables/summary_stats/oa_tab}

\subsection{Covariates \& Confounders}
We have included various demographic and socioeconomic covariates variables to serve as control variables. The health literature suggest that  certain socio-demographic and health behaviour variables are risk factors to heart disease and confounders in the OA-heart disease relationship. The socio-demographic variables in this study are age, sex, body mass index (BMI), ethnicity, geography, education and household income. The health behaviour variables in this study are physical activity, hypertension, access to a doctor, smoking status, diabetes and drinking habits. Tables \ref{socio_demogr_cov} and \ref{health_behav_cov} summarizes the health behaviour and socio-demographic covariates. 
