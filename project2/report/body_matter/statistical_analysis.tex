\section{Statistical Analysis}
In this analysis we use two models to determine the "popularity" of TED talks using the two responses; average views per day and the composite popularity score. To predict the average views per day, which is a count variable, we use a Poisson regression. To predict the composite popularity score we use a linear regression. We then see how these models perform when accounting for variation in time and talk themes using mixed models. 

\subsection{Poisson Regression on Average Views Per Day}
To predict the average views per day, which is a count variable, we use a Poisson regression with a log link. 
\begin{align}
\begin{split}
log(\text{Video $i$ Average Num.Views/Day})&=\text{Video Duration}_i+\text{Num. Speakers}_i\\
&+\text{Film Age}_i+\text{Title Length}_i\\
&+\text{Video Themes}_i+\text{Titles Content}_i\\
&+\text{Video Age Group}_i+\epsilon_i\\
\end{split}
\label{poisson_simple_eqn}
\end{align}
\begin{equation*}
i=\{1,..,2550\}
\end{equation*}
This model indicates the log of the average number of views a video will obtain on any given day given the predictors.
\subsection{Linear Regression on Composite Popularity Score}
To predict the popularity, which is a normally distributed number, we use a Linear regression model.
\begin{align}
	\begin{split}
	y_{ij}&=\beta_0+\b_0j\beta_{1}\text{spiciness}_{ij}+(\beta_{2}+b_{1j})\text{taco_type}_{ij}+(\beta_{3}+b_{2j})\text{meat_amount}_{ij}+\varepsilon_{ij}\\
	\end{split}
\label{big tings}
\end{align}
\begin{equation*}
i=\{1,...,n_j\}
j=\{1,...,j\}
\end{equation*}

This model indicates the popularity score a video will obtain given the predictors.

\subsection{Mixed Models}
As these videos vary greatly in when they were published/created and their themes, we model how these differences might affect the response variables using Mixed Models for the previous regressions. 
\subsubsection{Time Variation}
To determine whether the characteristics that predict popularity change over time, we define time into two groups, old (videos published prior to 2010) and new (after 2010). 
We would like to see how the relationship changes when TED talks had a surge in videos produced during 2010 that might be attributed to a change in an underlying demand for TED talk videos such as a new generation having access to the internet for example. 
As such, we use a Poisson mixed-effects model with random slope and intercept.
We chose to include random intercepts and slopes. 
We use random intercepts because we believe that the viewers from the early 1990s are different from viewers from 2017. Indeed, viewers of TED talk videos in 1990 when personal computers where not readily available were most likely from a higher socio-demographic population than the average viewer in 2017 when the vast majority of individuals have personal computers. As such, 'old' and 'new' videos will have different intercepts (different average popularity or average views/day).
Further, we include random slopes to account for potential differences in populations that might have different video viewing behaviors. Indeed, an individual in the early 2000s with a computer was most likely more educated (due to the general lack of widespread computers they would need it for very specific reasons like work/school but not for pleasure, generally speaking) than the average computer user in 2017 (where everyone owns a computer for both pleasure and work), and as such the former group might rate a longer video better than the latter group due to differences in attention span, for example. 

\subsubsection{Poisson Mixed Model Regression on Average Views Per Day: Time Variation}
\subsubsection{Linear Mixed Model Regression on Composite Popularity Score: Time Variation}

\subsubsection{Themes Variation}


\subsubsection{Poisson Mixed Model Regression on Average Views Per Day: Theme Variation}
\subsubsection{Linear Mixed Model Regression on Composite Popularity Score: Theme Variation}


Similarly, we apply Mixed Models to the Linear regression.


Poisson regression.


As such   odds of heart disease given having OA (i.e less access to Western/quality medical care that could prevent treatment for OA in the northern region). For simplicity, we denote $Demographic$ as the variable that encompasses all socio-demographic and health behaviour variables. $logit(P(HD_{ij}))$ denotes the odds of an individual having heart disease.