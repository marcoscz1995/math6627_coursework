\begin{frame}{Results: Linear Regressions}
    
% Table created by stargazer v.5.2.2 by Marek Hlavac, Harvard University. E-mail: hlavac at fas.harvard.edu
% Date and time: Tue, Mar 10, 2020 - 10:49:22 PM
% Requires LaTeX packages: dcolumn 
\begin{longtable}{@{\extracolsep{5pt}}lD{.}{.}{-3} D{.}{.}{-3} } 
  \caption{Poisson and Linear Regression} 
  \label{} 
\\[-1.8ex]\hline 
\endhead
\hline \\[-1.8ex] 
 & \multicolumn{2}{c}{\textit{Dependent variable:}} \\ 
\cline{2-3} 
\\[-1.8ex] & \multicolumn{1}{c}{avg\_views\_per\_day} & \multicolumn{1}{c}{popularity} \\ 
 & \multicolumn{1}{c}{Poisson} & \multicolumn{1}{c}{Linear} \\ 
\hline \\[-1.8ex] 
 Duration & 1.732^{***}$ $(0.008) & 0.258^{***}$ $(0.058) \\ 
  Num. Speaker & -0.823^{***}$ $(0.009) & -0.035$ $(0.079) \\ 
  Film Age in Days & -14.052^{***}$ $(0.014) & -1.433^{***}$ $(0.081) \\ 
  Title Label: Life & 0.158^{***}$ $(0.002) & 0.005$ $(0.023) \\ 
  Title Label: New & -0.144^{***}$ $(0.003) & -0.002$ $(0.024) \\ 
  Title Label: World & -0.096^{***}$ $(0.002) & -0.043^{***}$ $(0.015) \\ 
  Title Length & 0.196^{***}$ $(0.004) & 0.216^{***}$ $(0.031) \\ 
  Theme Label: Business & -0.132^{***}$ $(0.002) & -0.006$ $(0.023) \\ 
  Theme Label: Culture & -0.306^{***}$ $(0.002) & -0.041^{**}$ $(0.019) \\ 
  Theme Label: Design & -0.511^{***}$ $(0.002) & -0.052^{**}$ $(0.020) \\ 
  Theme Label: Energy & -0.579^{***}$ $(0.004) & -0.082^{***}$ $(0.031) \\ 
  Theme Label: Global & -0.888^{***}$ $(0.003) & -0.079^{***}$ $(0.020) \\ 
  Theme Label: Health & -0.681^{***}$ $(0.003) & -0.002$ $(0.022) \\ 
  Theme Label: Music & -0.477^{***}$ $(0.003) & -0.118^{***}$ $(0.026) \\ 
  Theme Label: Science & -0.655^{***}$ $(0.002) & -0.030$ $(0.020) \\ 
  Theme Label: Social & -0.493^{***}$ $(0.002) & -0.013$ $(0.022) \\ 
  Video Age Group: Old & 0.514^{***}$ $(0.002) & 0.035^{***}$ $(0.013) \\ 
  Intercept & 8.706^{***}$ $(0.003) & 1.852^{***}$ $(0.024) \\ 
 \hline \\[-1.8ex] 
Observations & \multicolumn{1}{c}{2,550} & \multicolumn{1}{c}{2,550} \\ 
R$^{2}$ &  & \multicolumn{1}{c}{0.296} \\ 
Adjusted R$^{2}$ &  & \multicolumn{1}{c}{0.291} \\ 
Akaike Inf. Crit. & \multicolumn{1}{c}{2,217,004.000} &  \\ 
\hline 
\hline \\[-1.8ex] 
\textit{Note:}  & \multicolumn{2}{r}{$^{*}$p$<$0.1; $^{**}$p$<$0.05; $^{***}$p$<$0.01} \\ 
\end{longtable} 

\end{frame}
\begin{frame}{Results: Linear and Regressions}
	
% Table created by stargazer v.5.2.2 by Marek Hlavac, Harvard University. E-mail: hlavac at fas.harvard.edu
% Date and time: Tue, Mar 10, 2020 - 11:24:37 PM
% Requires LaTeX packages: dcolumn 
\begin{longtable}{@{\extracolsep{5pt}}lD{.}{.}{-3} D{.}{.}{-3} } 
  \label{simple_results} 
\endhead
\hline \\[-1.8ex] 
\cline{2-3} 
\\[-1.8ex] & \multicolumn{1}{c}{avg\_views\_per\_day} & \multicolumn{1}{c}{popularity} \\ 
 & \multicolumn{1}{c}{Poisson} & \multicolumn{1}{c}{Linear} \\ 
\hline \\[-1.8ex] 
  Theme Label: Business & -0.135^{***}$ $(0.002) & -0.002$ $(0.022) \\ 
  Theme Label: Culture & -0.314^{***}$ $(0.002) & -0.043^{**}$ $(0.018) \\ 
  Theme Label: Design & -0.537^{***}$ $(0.002) & -0.057^{***}$ $(0.019) \\ 
  Theme Label: Energy & -0.589^{***}$ $(0.004) & -0.082^{***}$ $(0.029) \\ 
  Theme Label: Global & -0.845^{***}$ $(0.003) & -0.059^{***}$ $(0.019) \\ 
  Theme Label: Health & -0.692^{***}$ $(0.003) & -0.004$ $(0.021) \\ 
  Theme Label: Music & -0.523^{***}$ $(0.003) & -0.130^{***}$ $(0.025) \\ 
  Theme Label: Science & -0.669^{***}$ $(0.002) & -0.030$ $(0.019) \\ 
  Theme Label: Social & -0.552^{***}$ $(0.002) & -0.047^{**}$ $(0.021) \\ 
  Video Age Group: Old & -1.841^{***}$ $(0.006) & -0.317^{***}$ $(0.026) \\ 
  Film Age:Video Age Group: Old & 13.460^{***}$ $(0.031) & 2.434^{***}$ $(0.153) \\ 
  Intercept & 8.952^{***}$ $(0.003) & 2.038^{***}$ $(0.025) \\ 
 \hline \\[-1.8ex] 
Observations & \multicolumn{1}{c}{2,550} & \multicolumn{1}{c}{2,550} \\ 
R$^{2}$ &  & \multicolumn{1}{c}{0.360} \\ 
Adjusted R$^{2}$ &  & \multicolumn{1}{c}{0.356} \\ 
Akaike Inf. Crit. & \multicolumn{1}{c}{2,071,454.000} &  \\ 
\hline 
\hline \\[-1.8ex] 
\textit{Note:}  & \multicolumn{2}{r}{$^{*}$p$<$0.1; $^{**}$p$<$0.05; $^{***}$p$<$0.01} \\ 
\end{longtable} 

\end{frame}
\begin{frame}{Results: Linear and Regressions}

% Table created by stargazer v.5.2.2 by Marek Hlavac, Harvard University. E-mail: hlavac at fas.harvard.edu
% Date and time: Wed, Apr 01, 2020 - 01:05:29 AM
% Requires LaTeX packages: dcolumn 
\begin{longtable}{@{\extracolsep{5pt}}lD{.}{.}{-3} D{.}{.}{-3} } 
  \caption{Economic and Human Cost Linear Regression} 
  \label{results_table} 
\\[-1.8ex]\hline 
\endhead
\hline \\[-1.8ex] 
 & \multicolumn{2}{c}{\textit{Dependent variable:}} \\ 
\cline{2-3} 
\\[-1.8ex] & \multicolumn{1}{c}{NORMALIZED.TOTAL.COST} & \multicolumn{1}{c}{human\_cost\_comp\_score} \\ 
 & \multicolumn{1}{c}{Economic Cost(Millions)} & \multicolumn{1}{c}{Human Cost} \\ 
\hline \\[-1.8ex] 
   ON & -19.376$ $(43.658) & -0.008$ $(0.015) \\ 
   PE & -103.227$ $(68.038) & -0.006$ $(0.023) \\ 
   QC & -13.482$ $(46.461) & -0.012$ $(0.016) \\ 
   SK & -33.949$ $(48.191) & -0.005$ $(0.016) \\ 
   YT & -62.270$ $(88.978) & -0.001$ $(0.030) \\ 
  Year & 1.190^{**}$ $(0.465) & -0.0004^{***}$ $(0.0002) \\ 
  Intercept & -2,327.665^{**}$ $(925.329) & 0.887^{***}$ $(0.311) \\ 
 \hline \\[-1.8ex] 
Observations & \multicolumn{1}{c}{1,114} & \multicolumn{1}{c}{1,114} \\ 
R$^{2}$ & \multicolumn{1}{c}{0.037} & \multicolumn{1}{c}{0.697} \\ 
Adjusted R$^{2}$ & \multicolumn{1}{c}{0.007} & \multicolumn{1}{c}{0.688} \\ 
\hline 
\hline \\[-1.8ex] 
\textit{Note:}  & \multicolumn{2}{r}{$^{*}$p$<$0.1; $^{**}$p$<$0.05; $^{***}$p$<$0.01} \\ 
\end{longtable} 

\end{frame}

\begin{frame}{Results: Linear Regressions}
\begin{itemize}
	\item \textbf{Event Type}: Both models do not agree on any of the type of events.Recall, each $\beta_{event type}$ is the difference between the average response for that event type and the the average response for the base event type 'Avalanche'. Since all the events, except for winter storms, are not significant, this suggests that strongest predictors of costs is anything snow related. However, this is not the case for human costs. For human costs; cold and heat events are among the most impact full events.
\item \textbf{Earthquakes Magnitude}: Both models agree that earthquake magnitude is not significant.
\item \textbf{Events Duration}: Only the human costs model suggests that the event duration is significant and positive. We posit that many of the events occur over a short period of time, over a day or two, so durration is not a strong predictor for economic costs, however epidemics last a very long time and thus will incur more impacts to humans than psychical capital.
\end{itemize}
\end{frame}

\begin{frame}{Results: Linear Regressions}
\begin{itemize}
	\item \textbf{Num. Provinces Involved}: Only the human costs model indicates it as significant. Although what is interesting is that it suggests that as more provinces are involved, the human impact decreases. We posit that this might be because as more provinces are involved, the loss is "diluted" across the provinces.
	\item \textbf{Province of Event}: Both models find the provinces to not be significant, which suggests that losses are felt equally among provinces.
\end{itemize}
\end{frame}

\begin{frame}{Results: Linear Regressions}
\begin{itemize}

	\item \textbf{Year of Event}: What is interesting is that the economics model suggests that costs have increased over time whereas human costs have decreased over time. We posit that with advancements in science we are better equipped to forecast events and thus evacuate people thus reducing human losses. However, the increase in costs can be for several reasons; increases "strength" of natural disasters over time, increases in the cost of items damaged. A flood that wreaks havoc in 1920 Southern Ontario, even when adjusting for inflation, will incur less absolute economic costs than a flood in 2020 Southern Ontario.
	\item \textbf{Adjusted $R^2$}: The economic costs model has an adjusted $R^2$ of almost 0, where as the human costs is almost .7. This suggests that these variables do not predict the economic impacts well, where as it does well for the human impact.
\end{itemize}
\end{frame}

