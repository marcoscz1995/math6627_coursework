\begin{frame}{Response}
	In this analysis, we used two measures of natural disaster impacts; the economic and human costs.
\begin{itemize}
	\item We decided to separate these two predictors as we believe that these two variables are very different and no one measure can capture both of them adequately. Further, due to changes in the size of governments, demographics and science will cause natural disasters to effect each one differently. 
	\item For example, as science improve we are able to reduce biological pandemics which impacts human life more directly than economic costs; in contrast a flood that destroys an evacuated town (as we become better at predicting floods) will have more economic than human costs.

\end{itemize}
\end{frame}

\begin{frame}{Economic Impact}
We composed a composite popularity score using the following variables:
\begin{itemize}
\item The most quantifiable measurement of a disasters impact is its economic cost.
\item This includes damages to property and businesses, forgone wages, and increase in investments to replace damaged capital.
\item  As we transformed the estimated total cost variable to 2014 prices using the canadian CPI measure, we control for inflation which allows us to compare costs of each event across time.
\item If a natural event occurred in multiple provinces, then the total cost for the event in one of those provinces needs to be standardized by its population. The reason being that the impact of the natural disaster is not spread evenly across provinces and hence neither will the total costs in the data.

\end{itemize}
\end{frame}

\begin{frame}{Economic Impact}
\begin{itemize}
	\item For example, a flood that effects both Quebec and New Brunswick will incur different costs. Our reasoning is that since Quebec is a more capital intensive economy than New Brunswick, it will incur more costs in damaged physical capital, thus incurring a larger economic loss in absolute terms.
	\item Thus, the datasets does not reflect this asymmetry in losses and provides just the total Federal loss to the event and not the provincial loss.
	\item As such, we standardize the losses by multiplying the loss by each provinces proportional population size to get the cost per province.
\end{itemize}
\end{frame}

\begin{frame}{Human Impact: A Composite Score}
	\begin{itemize}
		\item The data set provides three variables that are related to the direct impact to humans; fatalities, injured/infected, evacuated and number of people that loose access to utility services. 
\item We transform all the human impact measures into percentages; we divide the total human loss by the sum of all provinces population involved to get what percentage of each provinces population is effected. We assume that every province involved suffers the same amount of human loss in percentages. 
\item By transforming the human loss into percentage this allows us to control for increases in population over time which makes comparing losses over time easier.

	\end{itemize}
	
\end{frame}

\begin{frame}{Human Impact: A Composite Score}
\begin{itemize}

	\item We composed a composite human impact score using these variables. We use equal weightings in the construction of the composite variable.
	\item Since all of these variables are on vastly different scales of magnitude, we normalize the variables and add them to create our composite human impact score. 
\end{itemize}

\end{frame}

\begin{frame}{Predictors}
\begin{itemize}
	\item \textbf{Event Type}: We include the event type as the main predictor as we predict that different types of events will result with differing effects on economic and human costs. For example, a biological pandemic will create more human loss than direct economic costs, whereas a flood will cause more economic damage than human loss.
	\item \textbf{Earthquakes Magnitude}: We assume that stronger earthquakes will cause greater economic and human loss.
	\item \textbf{Events Duration}: We predict that as an events duration increases, both measures will increase. 

\end{itemize}
\end{frame}

\begin{frame}{Predictors}
\begin{itemize}

	\item \textbf{Number of Provinces Involved}: We included this as we posit that as more provinces are involved the costs will increase as all levels of government are strained and not able to focus on mitigating costs. Indeed, as more provinces are involved, then it is likely that more government and hence more bureaucratic slog will slow down responses to crisis management.
	\item \textbf{Province of Event}: The data indicates in which provinces an event occurred. We would like to see how costs are ranked by provinces.
	\item \textbf{Year of Event}: We would like to see how costs evolve over time. 
\end{itemize}
\end{frame}

\begin{frame}{}
	% latex table generated in R 3.4.4 by xtable 1.8-4 package
% Tue Mar 10 19:42:42 2020
\begingroup\footnotesize
\begin{longtable}{ll|rrr}
 \textbf{Variable} & \textbf{Levels} & $\mathbf{n}$ & $\mathbf{\%}$ & $\mathbf{\sum \%}$ \\ 
  \hline
Title Label & future & 2197 & 86.2 & 86.2 \\ 
   & life & 85 & 3.3 & 89.5 \\ 
   & new & 73 & 2.9 & 92.3 \\ 
   & world & 195 & 7.7 & 100.0 \\ 
   \hline
 & all & 2550 & 100.0 &  \\ 
   \hline
\hline
Video Theme & brain & 147 & 5.8 & 5.8 \\ 
   & business & 184 & 7.2 & 13.0 \\ 
   & culture & 605 & 23.7 & 36.7 \\ 
   & design & 329 & 12.9 & 49.6 \\ 
   & energy & 64 & 2.5 & 52.1 \\ 
   & global & 354 & 13.9 & 66.0 \\ 
   & health & 192 & 7.5 & 73.5 \\ 
   & music & 118 & 4.6 & 78.2 \\ 
   & science & 346 & 13.6 & 91.7 \\ 
   & social & 211 & 8.3 & 100.0 \\ 
   \hline
 & all & 2550 & 100.0 &  \\ 
   \hline
\hline
Video Age Label & new & 1711 & 67.1 & 67.1 \\ 
   & old & 839 & 32.9 & 100.0 \\ 
   \hline
 & all & 2550 & 100.0 &  \\ 
   \hline
\hline
\hline
\caption{categorical} 
\label{cat}
\end{longtable}
\endgroup

\end{frame}


