\section{Conclusion}
In this analysis we used data from the CDD and web scraped data from the Demographics Wikipedia pages for each province to determine that impacts of natural disasters in Canadian provinces. We measured impact by using an economics cost and a composite human costs score that encompasses several intuitive measures of human loss.
We used linear mixed models with random intercepts and slopes to exploit variation in events across time and provinces. We found that the majority of variation in the data can be explained by variation across time and provinces. In particular, we find that there is weak evidence to support random intercepts for both cost models with time and provincial random effects. Further, we find that there is little evidence that suggests that the variables that predict costs vary by provinces, however we find that the random slopes vary over time. We posit that although all provinces are effected the same by natural disasters, the impacts will increase for all provinces over time.
We also would like to point out that we are unable to determine what causes the increase in costs over time. This could be due to general increases in the amount of physical capital that can be damaged, a measure that adjusting for inflation cannot capture, or whether this is because of the "strength" of natural disasters is increasing. Further analysis would benefit from including the magnitudes of each type of event. 