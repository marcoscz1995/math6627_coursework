\section{Model Selection}
We now test the random intercepts and random slopes for each model. That is, we test whether there exists difference in the average responses across time and provinces, and whether there exists differences in how the predictors effect the response across time and provinces.
To compare a generalized mixed model (GLMM) without a random component with a GLMM with a random component in R, we had to use the GLM package for the nested model and LMER4 for the full model as LMER4 does not allow models to not have a random component. Evidently, to compare the nested and full model with random intercepts, we used AIC as our criterion because the ANOVA function in R is not able to compare models from the LMER4 and GLM packages. Table \ref{aic} summarizes the results of testing the random intercept.

\begin{itemize}
	\item \textbf{Economic Cost Model with Random Intercept for Provinces}: The AIC for the mixed model with random intercept is lower than the null, so it is preferred. This suggest that there exists differences in the average popularity score across themes.
	\item \textbf{Economic Cost Model with Random Intercept for Time}: The AIC for the mixed model with random intercept is lower than the null, so it is preferred. This suggest that there exists differences in the average popularity score across time, that is for videos published before and after 2010.
	\item \textbf{Human Cost Model with Random Intercept for Provinces}: The AIC for the mixed model with random intercept is lower than the null, so it is preferred. This suggest that there exists differences in the average avg. views per day across themes.
	\item \textbf{Human Cost Model with Random Intercept for Time}: The AIC for the mixed model with random intercept is lower than the null, so it is preferred. This suggest that there exists differences in the average avg. views per day across time, that is for videos published before and after 2010.
\end{itemize}

\begin{table}[ht]
	   \caption{Testing Random Intercept} 
	\label{aic}
\centering
\begin{tabular}{rrr}
  \hline
 & df & AIC \\ 
  \hline
Economic Cost Povince null & 22.00 & 47151.83 \\ 
Economic Cost Povince full & 23.00 & 46376.05 \\ 
Economic Cost Time null & 22.00 & 47151.83 \\ 
Economic Cost Time full & 23.00 & 46357.20 \\ 
Human Cost Province null & 22.00 & -1447.57 \\ 
Human Cost Province full & 23.00 & -1307.21 \\ 
Human Cost Time null & 22.00 & -1447.57 \\  
Human Cost Time full & 23.00 & -1394.44 \\ 
   \hline
\end{tabular}
\end{table}

Similarly, we test the random slopes. We test these random slopes using $\chi^2$ model selection. Table \ref{anova}. summarizes the results.
\begin{itemize}
	\item \textbf{Economic Cost Model with Random Slope for Provinces}: The p-value is very small, so the model without random slope is not an adequate simplification of the full model; the preferred model includes the random intercept. 
	This suggest that there exists differences in how the predictors effect popularity score across themes.
	\item \textbf{Economic Cost Model with Random Slope for Time}: The p-value is very small, so the model without random slope is not an adequate simplification of the full model; the preferred model includes the random intercept. This suggest that there exists differences in how the predictors effect popularity score across time, that is for videos published before and after 2010.
	\item \textbf{Human Cost Model with Random Slope for Provinces}: The p-value is very small, so the model without random slope is not an adequate simplification of the full model; the preferred model includes the random intercept. This suggest that there exists differences in how the predictors effect avg. views per day across themes.
	\item \textbf{Human Cost Model with Random Slope for Time}: The p-value is very small, so the model without random slope is not an adequate simplification of the full model; the preferred model includes the random intercept. This suggest that there exists differences in how the predictors effect avg. views per day across time, that is for videos published before and after 2010.
\end{itemize}
% latex table generated in R 3.4.4 by xtable 1.8-4 package
% Mon Mar  9 21:33:01 2020
\begin{table}[ht]
	   \caption{Testing Random Slopes} 
	\label{anova}
\centering
\begin{tabular}{lrrrrr}
  \hline
 & Df & Chisq & Chi Df & Pr($>$Chisq) \\ 
  \hline
Economic Cost Povince simple & 23&  &  &  \\ 
Economic Cost Povince full & 253 & 0.00 & 230 & 1.0000 \\ 
Economic Cost Time simple & 23 &  &  &  \\ 
Economic Cost Time full & 253 & 935.65 & 230 & 0.0000 \\ 
Human Cost Province simple & 23 &  &  &  \\ 
Human Cost Province full & 253 & 47.41 & 230 & 1.0000 \\ 
Human Cost Time simple & 23 &  &  &  \\ 
Human Cost Time full & 253 & 1719.02 & 230 & 0.0000 \\ 
   \hline

\end{tabular}
\end{table}

Clearly, the best models include both random intercepts and slopes for both Poisson and Linear models. Which suggests that a lot of the variation in the data can be explained by timing of when a video is published and its theme rather than the actual form/content. The main predictors in these models are the talks duration, talks film age in days since published, and the talks age group (when the talk was published) are the strongest predictors of popularity. This suggests that the best way to create a popular video is by frequently publishing long videos, regardless of the content. Popularity seems to be a function of quantity not quality.
















