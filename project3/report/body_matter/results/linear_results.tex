\section{Results}
\subsection{Linear Regression Results}
We first look at the result from the linear regressions on the economic and human impacts of natural disasters without any random components. Table \ref{results_table} summarizes the results for the linear regressions for the economic and human costs.


% Table created by stargazer v.5.2.2 by Marek Hlavac, Harvard University. E-mail: hlavac at fas.harvard.edu
% Date and time: Tue, Mar 10, 2020 - 10:49:22 PM
% Requires LaTeX packages: dcolumn 
\begin{longtable}{@{\extracolsep{5pt}}lD{.}{.}{-3} D{.}{.}{-3} } 
  \caption{Poisson and Linear Regression} 
  \label{} 
\\[-1.8ex]\hline 
\endhead
\hline \\[-1.8ex] 
 & \multicolumn{2}{c}{\textit{Dependent variable:}} \\ 
\cline{2-3} 
\\[-1.8ex] & \multicolumn{1}{c}{avg\_views\_per\_day} & \multicolumn{1}{c}{popularity} \\ 
 & \multicolumn{1}{c}{Poisson} & \multicolumn{1}{c}{Linear} \\ 
\hline \\[-1.8ex] 
 Duration & 1.732^{***}$ $(0.008) & 0.258^{***}$ $(0.058) \\ 
  Num. Speaker & -0.823^{***}$ $(0.009) & -0.035$ $(0.079) \\ 
  Film Age in Days & -14.052^{***}$ $(0.014) & -1.433^{***}$ $(0.081) \\ 
  Title Label: Life & 0.158^{***}$ $(0.002) & 0.005$ $(0.023) \\ 
  Title Label: New & -0.144^{***}$ $(0.003) & -0.002$ $(0.024) \\ 
  Title Label: World & -0.096^{***}$ $(0.002) & -0.043^{***}$ $(0.015) \\ 
  Title Length & 0.196^{***}$ $(0.004) & 0.216^{***}$ $(0.031) \\ 
  Theme Label: Business & -0.132^{***}$ $(0.002) & -0.006$ $(0.023) \\ 
  Theme Label: Culture & -0.306^{***}$ $(0.002) & -0.041^{**}$ $(0.019) \\ 
  Theme Label: Design & -0.511^{***}$ $(0.002) & -0.052^{**}$ $(0.020) \\ 
  Theme Label: Energy & -0.579^{***}$ $(0.004) & -0.082^{***}$ $(0.031) \\ 
  Theme Label: Global & -0.888^{***}$ $(0.003) & -0.079^{***}$ $(0.020) \\ 
  Theme Label: Health & -0.681^{***}$ $(0.003) & -0.002$ $(0.022) \\ 
  Theme Label: Music & -0.477^{***}$ $(0.003) & -0.118^{***}$ $(0.026) \\ 
  Theme Label: Science & -0.655^{***}$ $(0.002) & -0.030$ $(0.020) \\ 
  Theme Label: Social & -0.493^{***}$ $(0.002) & -0.013$ $(0.022) \\ 
  Video Age Group: Old & 0.514^{***}$ $(0.002) & 0.035^{***}$ $(0.013) \\ 
  Intercept & 8.706^{***}$ $(0.003) & 1.852^{***}$ $(0.024) \\ 
 \hline \\[-1.8ex] 
Observations & \multicolumn{1}{c}{2,550} & \multicolumn{1}{c}{2,550} \\ 
R$^{2}$ &  & \multicolumn{1}{c}{0.296} \\ 
Adjusted R$^{2}$ &  & \multicolumn{1}{c}{0.291} \\ 
Akaike Inf. Crit. & \multicolumn{1}{c}{2,217,004.000} &  \\ 
\hline 
\hline \\[-1.8ex] 
\textit{Note:}  & \multicolumn{2}{r}{$^{*}$p$<$0.1; $^{**}$p$<$0.05; $^{***}$p$<$0.01} \\ 
\end{longtable} 


What is interesting is that the human costs model Poisson model kept everything as significant whereas the Linear model removed the title labels Life and New and the theme labels science and social. Further, both models agree on all signs and rankings in the case of categorical variables which adds to the robustness of these models ability to predict popularity. 
Note, for this analysis we use terms such as increase or decrease for simplicity.

\begin{itemize}
	\item \textbf{Event Type}: Both models do not agree on any of the type of events. What is interesting is that the only relavent event for economic costs is 
	\item \textbf{Earthquakes Magnitude}: We assume that stronger earthquakes will cause greater economic and human loss.
	\item \textbf{Events Duration}: We predict that as an events duration increases, both measures will increase. 
	\item \textbf{Province of Event}: The data indicates in which provinces an event occurred. We would like to see how costs are ranked by provinces.
	\item \textbf{Year of Event}: We would like to see how costs evolve over time. 
\end{itemize}


\begin{itemize}
	\item \textbf{Duration}: Both models report an increase in the response, which is counter intuitive as we initially believed that viewers would be turned off from longer videos.
	\item \textbf{Number of Speaker}: Both models reported a decrease in the response, however the linear model did not consider it significant. This goes against our initial beliefs that more speakers would increase the likelihood of a viewer seeing a speaker they enjoy. We posit that this could be because more speakers could raise the chances that a viewer will dislike one of the speakers and not watch the video.
	\item \textbf{Film Age}: Both models report significant decreases in the response, which agrees with our initial hypothesis that viewers prefer videos that have been published more recently. 
	\item \textbf{Video Age Group}: Both models report decreases in the response. This supports our initial hypothesis that viewers prefer videos that have been produced more recently.
	\item \textbf{Film Age:Video Age Group: Old}: We also model the interaction between these two terms to see how talks produced prior to 2010 but published after 2010 performs. Both models report an increase in the responses, which suggests that an 'old' video but published recently will perform just as well as a 'new' video published recently. This suggest that the age of the video in days since posted is a stronger predictor for success than the actual content of the talk. 
	\item \textbf{Title Length}: Both models report increases in the response and are significant. This is surprising as we thought verbose titles would discourage viewers from watching the video. 
	\item \textbf{Themes Label}: Both models report all the labels as significant and negative. Recall, each $\beta_{theme}$ is the difference between the average response for that theme and the the average response for the base theme 'Brain'. Since all the themes are negative, this suggests that the best theme for talks is about the Brain. Further, as these coefficients act as rankings between the levels where the values closest to zero suggest higher popularity. Indeed, we see that 'Business' and 'Culture' are among the more popular themes while 'Global' and 'Health' are among the least popular. These results are true for both models
	\item \textbf{Title Label}: Using the same logic as above, we see that both models agree that titles regarding 'World' and 'New' are the least popular, that the base topic of 'Future' is the second most popular and 'Life' is the most popular. Both models only agree on the significance of 'World' which suggests that this topics popularity can be accurately predicted unlike the others.
\end{itemize}

\subsection{Mixed Models Results}
We now look at the result from the mixed models with random slope and intercept. Table \ref{mixed_results_table} summarizes the results.
\section{Results with Mixed Models}
\subsection{Poisson \& Linear Mixed-Effects}
We now look at the result from the Poisson and Linear mixed models with random slope and intercept. Table \ref{mixed_results_table} summarizes the results.

% Table created by stargazer v.5.2.2 by Marek Hlavac, Harvard University. E-mail: hlavac at fas.harvard.edu
% Date and time: Tue, Mar 10, 2020 - 12:53:56 AM
% Requires LaTeX packages: dcolumn 
\begin{longtable}{@{\extracolsep{5pt}}lD{.}{.}{-3} D{.}{.}{-3} D{.}{.}{-3} D{.}{.}{-3} } 
  \caption{Poisson and Normal Mixed-Effects with Random Slope and Intercept} 
  \label{mixed_results_table} 
\\[-1.8ex]\hline 
\endhead
\hline \\[-1.8ex] 
 & \multicolumn{4}{c}{\textit{Dependent variable:}} \\ 
\cline{2-5} 
\\[-1.8ex] & \multicolumn{2}{c}{avg\_views\_per\_day} & \multicolumn{2}{c}{popularity} \\ 
 & \multicolumn{1}{c}{Poisson Themes} & \multicolumn{1}{c}{Poisson Times} & \multicolumn{1}{c}{Normal Themes} & \multicolumn{1}{c}{Normal Times} \\ 
\hline \\[-1.8ex] 
 Duration & 1.976^{***}$ $(0.490) & 1.976^{***}$ $(0.490) & 0.399^{***}$ $(0.136) & 0.382^{***}$ $(0.128) \\ 
  Title Length & 0.155$ $(0.169) & 0.155$ $(0.169) & 0.178^{***}$ $(0.050) & 0.197^{***}$ $(0.030) \\ 
  Title Label: Life & 0.064$ $(0.093) & 0.064$ $(0.093) & 0.023$ $(0.032) & 0.006$ $(0.022) \\ 
  Title Label: New & -0.166$ $(0.147) & -0.166$ $(0.147) & -0.0002$ $(0.027) & -0.003$ $(0.024) \\ 
  Title Label: World & -0.129^{**}$ $(0.054) & -0.129^{**}$ $(0.054) & -0.048^{**}$ $(0.023) & -0.043^{***}$ $(0.015) \\ 
  Num. Speaker & -0.689$ $(0.755) & -0.689$ $(0.755) & -0.047$ $(0.124) & -0.048$ $(0.123) \\ 
  Film Age & -12.327^{***}$ $(0.698) & -12.327^{***}$ $(0.698) & -1.330^{***}$ $(0.145) & -1.309^{***}$ $(0.131) \\ 
  Intercept & 8.105^{***}$ $(0.104) & 8.105^{***}$ $(0.104) & 1.797^{***}$ $(0.035) & 1.791^{***}$ $(0.034) \\ 
 \hline \\[-1.8ex] 
Observations & \multicolumn{1}{c}{2,550} & \multicolumn{1}{c}{2,550} & \multicolumn{1}{c}{2,550} & \multicolumn{1}{c}{2,550} \\ 
Log Likelihood & \multicolumn{1}{c}{-1,075,431.000} & \multicolumn{1}{c}{-1,075,431.000} & \multicolumn{1}{c}{436.955} & \multicolumn{1}{c}{429.853} \\ 
Akaike Inf. Crit. & \multicolumn{1}{c}{2,150,950.000} & \multicolumn{1}{c}{2,150,950.000} & \multicolumn{1}{c}{-783.910} & \multicolumn{1}{c}{-821.705} \\ 
Bayesian Inf. Crit. & \multicolumn{1}{c}{2,151,207.000} & \multicolumn{1}{c}{2,151,207.000} & \multicolumn{1}{c}{-520.937} & \multicolumn{1}{c}{-710.672} \\ 
\hline 
 & \multicolumn{4}{c}{\textit{Random Effects Variance:}} \\ 
\hline
Intercept & 0.10631  & 1.8097 & 0.0093 & 0.0093\\ 
Duration & 2.22376  & 0.04407 & 0.1414 & 0.1413\\ 
Num. Speaker & 4.80851   & 0.106 &  0.0833 & 0.0833\\ 
Film Age & 4.09735  & 15.22 & 0.1711  & 0.1711\\ 
Title Length & 0.27191   & 0.728 & 0.0144 & 0.0141\\ 
Title Label: Life & 0.08319  & 1.063 & 0.0043 & 0.0043\\ 
Title Label: New & 0.2113  & 0.79156 & 0.00126  & 0.00124\\ 
Title Label: World & 0.0276 & 0.3207 & 0.00244 & 0.0401\\ 
\hline \\[-1.8ex] 

\hline \\[-1.8ex] 
Observations & \multicolumn{1}{c}{2,550} & \multicolumn{1}{c}{2,550} & \multicolumn{1}{c}{2,550} & \multicolumn{1}{c}{2,550} \\ 
Log Likelihood & \multicolumn{1}{c}{-1,075,431.000} & \multicolumn{1}{c}{-1,075,431.000} & \multicolumn{1}{c}{436.955} & \multicolumn{1}{c}{429.853} \\ 
Akaike Inf. Crit. & \multicolumn{1}{c}{2,150,950.000} & \multicolumn{1}{c}{2,150,950.000} & \multicolumn{1}{c}{-783.910} & \multicolumn{1}{c}{-821.705} \\ 
Bayesian Inf. Crit. & \multicolumn{1}{c}{2,151,207.000} & \multicolumn{1}{c}{2,151,207.000} & \multicolumn{1}{c}{-520.937} & \multicolumn{1}{c}{-710.672} \\ 
\textit{Note:}  & \multicolumn{4}{r}{$^{*}$p$<$0.1; $^{**}$p$<$0.05; $^{***}$p$<$0.01} \\ 
\end{longtable} 

Since mixed models will produce the same coefficients as its non mixed counterpart we skip the description of the coefficients. Further, all the models have very similar coefficients as their non mixed counterparts except some ar slightly different which could be because of convergence issues with the lmer package. The interpretations and findings from the previous section hold the same. \ref{simple_results}.
\subsubsection{Model Selection}
We now test the random intercepts and random slopes. That is, we test whether there exists difference in the average responses across time and themes, and whether there exists differences in how the predictors effect the response across time and themes.
To compare a generalized mixed model (GLMM) without a random component with a GLMM with a random component in R, we had to use the GLM package for the nested model and LMER4 for the full model as LMER4 does not allow models to not have a random component. Evidently, to compare the nested and full model with random intercepts, we used AIC as our criterion because the ANOVA function in R is not able to compare models from the LMER4 and GLM packages. Table \ref{aic} summarizes the results of testing the random intercept.

\begin{table}[ht]
	   \caption{Testing Random Intercept} 
	\label{aic}
\centering
\begin{tabular}{rrr}
  \hline
 & df & AIC \\ 
  \hline
Linear Popularity Themes null & 9.00 & -810.37 \\ 
Linear Popularity Themes full & 10.00 & -789.73 \\ 
Linear Popularity Times null & 9.00 & -810.37 \\ 
Linear Popularity Times full & 10.00 & -767.85 \\ 
Poisson Avg. Views/day Themes null & 8.00 & 2456258.87 \\ 
Poisson Avg. Views/day Themes full & 9.00 & 2262017.56 \\ 
Poisson Avg. Views/day Times null & 8.00 & 2456258.87 \\ 
Poisson Avg. Views/day Times full & 9.00 & 2418695.71 \\ 
   \hline
\end{tabular}
\end{table}

\begin{itemize}
	\item \textbf{Linear Model with Random Intercept for Themes}: The AIC for the mixed model with random intercept is lower than the null, so it is preferred. This suggest that there exists differences in the average popularity score across themes.
	\item \textbf{Linear Model with Random Intercept for Time}: The AIC for the mixed model with random intercept is lower than the null, so it is preferred. This suggest that there exists differences in the average popularity score across time, that is for videos published before and after 2010.
	\item \textbf{Poisson Model with Random Intercept for Themes}: The AIC for the mixed model with random intercept is lower than the null, so it is preferred. This suggest that there exists differences in the average avg. views per day across themes.
	\item \textbf{Poisson Model with Random Intercept for Time}: The AIC for the mixed model with random intercept is lower than the null, so it is preferred. This suggest that there exists differences in the average avg. views per day across time, that is for videos published before and after 2010.
\end{itemize}

Similarly, we test the random slopes. We test these random slopes using $\chi^2$ model selection. Table \ref{anova}. summarizes the results.
\begin{itemize}
	\item \textbf{Linear Model with Random Slope for Themes}: The p-value is very small, so the model without random slope is not an adequate simplification of the full model; the preferred model includes the random intercept. This suggest that there exists differences in how the predictors effect popularity score across themes.
	\item \textbf{Linear Model with Random Slope for Time}: The p-value is very small, so the model without random slope is not an adequate simplification of the full model; the preferred model includes the random intercept. This suggest that there exists differences in how the predictors effect popularity score across time, that is for videos published before and after 2010.
	\item \textbf{Poisson Model with Random Slope for Themes}: The p-value is very small, so the model without random slope is not an adequate simplification of the full model; the preferred model includes the random intercept. This suggest that there exists differences in how the predictors effect avg. views per day across themes.
	\item \textbf{Poisson Model with Random Slope for Time}: The p-value is very small, so the model without random slope is not an adequate simplification of the full model; the preferred model includes the random intercept. This suggest that there exists differences in how the predictors effect avg. views per day across time, that is for videos published before and after 2010.
\end{itemize}
% latex table generated in R 3.4.4 by xtable 1.8-4 package
% Mon Mar  9 21:33:01 2020
\begin{table}[ht]
	   \caption{Testing Random Slopes} 
	\label{anova}
\centering
\begin{tabular}{lrrrrr}
  \hline
 & Df & Chisq & Chi Df & Pr($>$Chisq) \\ 
  \hline
Economic Cost Povince simple & 23&  &  &  \\ 
Economic Cost Povince full & 253 & 0.00 & 230 & 1.0000 \\ 
Economic Cost Time simple & 23 &  &  &  \\ 
Economic Cost Time full & 253 & 935.65 & 230 & 0.0000 \\ 
Human Cost Province simple & 23 &  &  &  \\ 
Human Cost Province full & 253 & 47.41 & 230 & 1.0000 \\ 
Human Cost Time simple & 23 &  &  &  \\ 
Human Cost Time full & 253 & 1719.02 & 230 & 0.0000 \\ 
   \hline

\end{tabular}
\end{table}

Clearly, the best models include both random intercepts and slopes for both Poisson and Linear models. Which suggests that a lot of the variation in the data can be explained by timing of when a video is published and its theme rather than the actual form/content. The main predictors in these models are the talks duration and age (in terms of how recent since it has been published). Which suggests that the best way to create a popular video is by having a long video and re-uploading it to escape the drop in popularity from aging.


Due to computational restrictions we were unable to run mixed models with both time and themes as random components.














Since mixed models will produce the same coefficients as its non mixed counterpart we skip the description of the coefficients. Further, all the models have very similar coefficients as their non mixed counterparts except some are slightly different which could be because of convergence issues with the lmer package. The interpretations and findings from the previous section hold the same.