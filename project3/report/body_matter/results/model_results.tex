\section{Results}
We now look at the result from the mixed models regressions on the economic and human impacts of natural disasters with variation by time and province with random intercepts and slopes. Table \ref{results_table} summarizes the results for the linear regressions for the economic and human costs.


% Table created by stargazer v.5.2.2 by Marek Hlavac, Harvard University. E-mail: hlavac at fas.harvard.edu
% Date and time: Tue, Mar 10, 2020 - 10:49:22 PM
% Requires LaTeX packages: dcolumn 
\begin{longtable}{@{\extracolsep{5pt}}lD{.}{.}{-3} D{.}{.}{-3} } 
  \caption{Poisson and Linear Regression} 
  \label{} 
\\[-1.8ex]\hline 
\endhead
\hline \\[-1.8ex] 
 & \multicolumn{2}{c}{\textit{Dependent variable:}} \\ 
\cline{2-3} 
\\[-1.8ex] & \multicolumn{1}{c}{avg\_views\_per\_day} & \multicolumn{1}{c}{popularity} \\ 
 & \multicolumn{1}{c}{Poisson} & \multicolumn{1}{c}{Linear} \\ 
\hline \\[-1.8ex] 
 Duration & 1.732^{***}$ $(0.008) & 0.258^{***}$ $(0.058) \\ 
  Num. Speaker & -0.823^{***}$ $(0.009) & -0.035$ $(0.079) \\ 
  Film Age in Days & -14.052^{***}$ $(0.014) & -1.433^{***}$ $(0.081) \\ 
  Title Label: Life & 0.158^{***}$ $(0.002) & 0.005$ $(0.023) \\ 
  Title Label: New & -0.144^{***}$ $(0.003) & -0.002$ $(0.024) \\ 
  Title Label: World & -0.096^{***}$ $(0.002) & -0.043^{***}$ $(0.015) \\ 
  Title Length & 0.196^{***}$ $(0.004) & 0.216^{***}$ $(0.031) \\ 
  Theme Label: Business & -0.132^{***}$ $(0.002) & -0.006$ $(0.023) \\ 
  Theme Label: Culture & -0.306^{***}$ $(0.002) & -0.041^{**}$ $(0.019) \\ 
  Theme Label: Design & -0.511^{***}$ $(0.002) & -0.052^{**}$ $(0.020) \\ 
  Theme Label: Energy & -0.579^{***}$ $(0.004) & -0.082^{***}$ $(0.031) \\ 
  Theme Label: Global & -0.888^{***}$ $(0.003) & -0.079^{***}$ $(0.020) \\ 
  Theme Label: Health & -0.681^{***}$ $(0.003) & -0.002$ $(0.022) \\ 
  Theme Label: Music & -0.477^{***}$ $(0.003) & -0.118^{***}$ $(0.026) \\ 
  Theme Label: Science & -0.655^{***}$ $(0.002) & -0.030$ $(0.020) \\ 
  Theme Label: Social & -0.493^{***}$ $(0.002) & -0.013$ $(0.022) \\ 
  Video Age Group: Old & 0.514^{***}$ $(0.002) & 0.035^{***}$ $(0.013) \\ 
  Intercept & 8.706^{***}$ $(0.003) & 1.852^{***}$ $(0.024) \\ 
 \hline \\[-1.8ex] 
Observations & \multicolumn{1}{c}{2,550} & \multicolumn{1}{c}{2,550} \\ 
R$^{2}$ &  & \multicolumn{1}{c}{0.296} \\ 
Adjusted R$^{2}$ &  & \multicolumn{1}{c}{0.291} \\ 
Akaike Inf. Crit. & \multicolumn{1}{c}{2,217,004.000} &  \\ 
\hline 
\hline \\[-1.8ex] 
\textit{Note:}  & \multicolumn{2}{r}{$^{*}$p$<$0.1; $^{**}$p$<$0.05; $^{***}$p$<$0.01} \\ 
\end{longtable} 


What is interesting is that the Poisson model kept everything as significant whereas the Linear model removed the title labels Life and New and the theme labels science and social. Further, both models agree on all signs and rankings in the case of categorical variables which adds to the robustness of these models ability to predict popularity. 
Note, for this analysis we use terms such as increase or decrease for simplicity.

\begin{itemize}
	\item \textbf{Event Type}: We include the event type as the main predictor as we predict that different types of events will result with differing effects on economic and human costs. For example, a biological pandemic will create more human loss than direct economic costs, whereas a flood will cause more economic damage than human loss.
	\item \textbf{Earthquakes Magnitude}: We assume that stronger earthquakes will cause greater economic and human loss.
	\item \textbf{Events Duration}: We predict that as an events duration increases, both measures will increase. 
\end{itemize}


\begin{itemize}
	\item \textbf{Duration}: Both models report an increase in the response, which is counter intuitive as we initially believed that viewers would be turned off from longer videos.
	\item \textbf{Number of Speaker}: Both models reported a decrease in the response, however the linear model did not consider it significant. This goes against our initial beliefs that more speakers would increase the likelihood of a viewer seeing a speaker they enjoy. We posit that this could be because more speakers could raise the chances that a viewer will dislike one of the speakers and not watch the video.
	\item \textbf{Film Age}: Both models report significant decreases in the response, which agrees with our initial hypothesis that viewers prefer videos that have been published more recently. 
	\item \textbf{Video Age Group}: Both models report decreases in the response. This supports our initial hypothesis that viewers prefer videos that have been produced more recently.
	\item \textbf{Film Age:Video Age Group: Old}: We also model the interaction between these two terms to see how talks produced prior to 2010 but published after 2010 performs. Both models report an increase in the responses, which suggests that an 'old' video but published recently will perform just as well as a 'new' video published recently. This suggest that the age of the video in days since posted is a stronger predictor for success than the actual content of the talk. 
	\item \textbf{Title Length}: Both models report increases in the response and are significant. This is surprising as we thought verbose titles would discourage viewers from watching the video. 
	\item \textbf{Themes Label}: Both models report all the labels as significant and negative. Recall, each $\beta_{theme}$ is the difference between the average response for that theme and the the average response for the base theme 'Brain'. Since all the themes are negative, this suggests that the best theme for talks is about the Brain. Further, as these coefficients act as rankings between the levels where the values closest to zero suggest higher popularity. Indeed, we see that 'Business' and 'Culture' are among the more popular themes while 'Global' and 'Health' are among the least popular. These results are true for both models
	\item \textbf{Title Label}: Using the same logic as above, we see that both models agree that titles regarding 'World' and 'New' are the least popular, that the base topic of 'Future' is the second most popular and 'Life' is the most popular. Both models only agree on the significance of 'World' which suggests that this topics popularity can be accurately predicted unlike the others.
\end{itemize}