\section{Statistical Analysis With Mixed Models}
As these events span over 100 years, and if we assume that climate change is causing more frequent and destructive natural events, then there is reason to believe that there exists variation in the data over time. Further, due to obvious geographical and social demographic difference among provinces, there is variation in the data among provinces. As such, we use mixed models to capture this time and space variation.

\subsection{Time Variation}
To determine whether the impact of natural disasters has changed over time, we divide time by the 12 decades from 1900 to 2014.
We chose to include random intercepts and slopes.
We use random intercepts because we believe that provinces evolve tremendously in every passing decade, in particular as global warming changes the environment, we posit that the average response variable will be different across time within a province. Also, due to advances in technology, such as river level controls in the Don Valley River, the average cost of these events will drop as we become better at managing crisis' over time.
Further, we include random slopes to account for potential differences in how the different types of events can effect the response variables over time. Indeed, a flood in the early 1900s was more destructive when houses were made of logs and hay, compared to todays more resilient buildings. 



\subsubsection{Linear Mixed Model Regression on Economic Cost: Time Variation}
We apply Mixed Models to the Linear regression to model the variation in time for the normalized economic cost. $\text{Economic Cost}_{ij}$ denotes the normalized economic impact.
\begin{align}
\begin{split}
\text{Economic Cost}_{ij}&=\beta_0+b_{0j}+(\beta_{1}+b_{1j})\text{Event Type}_{ij}+(\beta_{2}+b_{2j})\text{Event Duration}_{ij}+(\beta_{3}+b_{3j})\text{Magnitude}_{ij}\\ &+(\beta_{4}+b_{4j})\text{Num.Prov. Involved}_{ij}+\varepsilon_{ij}\\
\end{split}
\label{econ_time}
\end{align}
\begin{equation*}
i=\{1,...,n_j\},
j=\{1910,...,2008, 2014\}
\end{equation*}

In this model we have $b$'s as the random slope/intercept for $i$ observations from each time group $j$.
\subsubsection{Linear Mixed Model Regression on Composite Human Cost Response: Time Variation}
We apply Mixed Models to the Linear regression to model the variation in time for the composite human costs score $\text{Human Impact}_{ij}$ denotes the human costs score.
\begin{align}
\begin{split}
\text{Human Cost}_{ij}&=\beta_0+b_{0j}+(\beta_{1}+b_{1j})\text{Event Type}_{ij}+(\beta_{2}+b_{2j})\text{Event Duration}_{ij}+(\beta_{3}+b_{3j})\text{Magnitude}_{ij}\\ &+(\beta_{4}+b_{4j})\text{Num.Prov. Involved}_{ij}+\varepsilon_{ij}\\
\end{split}
\label{human_time.tex}
\end{align}
\begin{equation*}
i=\{1,...,n_j\},
j=\{1910,...,2008, 2014\}
\end{equation*}

In this model we have $b$'s as the random slope/intercept for $i$ observations from each time group $j$.

\subsection{Provincial Variation}
To determine whether the impact of natural disasters is different among provinces, we set the random levels as the 13 provinces and territories. We choose to use provinces over geographical regions (Maritimes, Prairies etc.) is because we believe that including more levels will allow a finer grain analysis, and also we posit that there is large variation within each region. For example, the prairies includes Alberta, Saskatchewan and Manitoba which despite sharing similar geographies and hence similar types of natural disasters, vary in their economies. These economics differences will lend themselves to difference in how to province prevents/mitigates natural disasters. Indeed, in oil rich Alberta, the government might have more resources to mitigate the harms of floods than Manitoba which in effect will results with lower economic and human losses.  
We chose to include random intercepts and slopes.
We use random intercepts because provinces vary greatly by geography and thus the type of events that effect them. More destructive events such as floods are concentrated in certain provinces, most notably the prairies. This concentration of events by geography will cause each province to have different average economic and human costs.
Further, we include random slopes to account for potential differences in how the different types of events can effect the response variables among provinces. Indeed, a flood in an area accustomed to floods with existing measures to cope will likely incur less damage than, say, Southern Ontario where floods are rare and will cause havoc as they are not prepared.

\subsubsection{Linear Mixed Model Regression on Economic Cost: Provincial Variation}
We apply Mixed Models to the Linear regression to model the variation among provinces for the normalized economic cost. $\text{Economic Cost}_{ij}$ denotes the normalized economic impact.
\begin{align}
\begin{split}
\text{Economic Cost}_{ij}&=\beta_0+b_{0j}+(\beta_{1}+b_{1j})\text{Event Type}_{ij}+(\beta_{2}+b_{2j})\text{Event Duration}_{ij}+(\beta_{3}+b_{3j})\text{Magnitude}_{ij}\\ &+(\beta_{4}+b_{4j})\text{Num.Prov. Involved}_{ij}+\varepsilon_{ij}\\
\end{split}
\label{econ_province}
\end{align}
\begin{equation*}
i=\{1,...,n_j\},
j=\{Alberta,...,Ontario, Quebec\}
\end{equation*}

In this model we have $b$'s as the random slope/intercept for $i$ observations from each time group $j$.
\subsubsection{Linear Mixed Model Regression on Composite Human Cost Response: Provincial Variation}
We apply Mixed Models to the Linear regression to model the variation among provinces for the composite human costs score $\text{Human Impact}_{ij}$ denotes the human costs score.
\begin{align}
\begin{split}
\text{Human Cost}_{ij}&=\beta_0+b_{0j}+(\beta_{1}+b_{1j})\text{Event Type}_{ij}+(\beta_{2}+b_{2j})\text{Event Duration}_{ij}+(\beta_{3}+b_{3j})\text{Magnitude}_{ij}\\ &+(\beta_{4}+b_{4j})\text{Num.Prov. Involved}_{ij}+\varepsilon_{ij}\\
\end{split}
\label{human_province}
\end{align}
\begin{equation*}
i=\{1,...,n_j\},
j=\{Alberta,...,Ontario, Quebec\}
\end{equation*}

In this model we have $b$'s as the random slope/intercept for $i$ observations from each time group $j$.































