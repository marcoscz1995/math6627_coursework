\section{Response and Predictors}
\subsection{Responses}
In this analysis, we used two measures of a disasters impact; an economic and human measure. We decided to separate these two predictors as we believe that these two variables are very different and no one measure can capture both of them adequately. Further, due to changes in the size of governments, demographics and science will cause natural disasters to effect each one differently. For example, as science improve we are able to reduce biological pandemics which impacts human life more directly than economic costs; in contrast a flood that destroys an evacuated town (as we become better at predicting floods) will have more economic than human costs.
Table \ref{numer} summarizes the two responses. One issue with the original datasets estimated human and economic costs is that if a natural event occurred in multiple provinces, then the total cost for the event in one of those provinces needs to be standardized by its population. The reason being that the impact of the natural disaster is not spread evenly across provinces and hence neither will the total costs in the data.
For example, a flood that effects both Quebec and New Brunswick will incur different costs. Our reasoning is that since Quebec has a higher population, more individual in absolute terms will be effected, and since Quebec is a more capital intensive economy than New Brunswick, it will incur more costs in damaged physical capital, thus incurring a larger economic loss in absolute terms. Thus, the datasets does not reflect this asymmetry in losses and provides just the total Federal loss to the event and not the provincial loss.
As such, we standardize the losses by multiplying the loss by each provinces proportional population size \footnote{each province population involved divided by the sum of all provinces population involved} to get the cost per province. Indeed, if New Brunswick is a quarter of Quebecs population, then it will incur a quarter of the costs and receive a quarter of all payments. Further, to account for inflation we measure all prices in the estimated total cost column in 2014 prices. \footnote{Recall that when joining the population and CDD data, we transformed the data to be within 1900 and 2014, so we use 2014 as the most recent year to compare all prices.}
However, for human impacts we transform them into percentages; we divide the total human loss by the sum of all provinces population involved to get what percentage of each provinces population was effected. We assume that every province involved suffers the same amount of human loss in percentages. By transforming the human loss into percentage this allows us to control for increases in population over time which makes comparing losses over time easier.

\subsubsection{Economic Impact}
The most quantifiable measurement of a disasters impact is its economic cost. This includes damages to property and businesses, forgone wages, and increase in investments to replace damaged capital. As we transformed the estimated total cost variable to 2014 prices using the canadian CPI measure, we control for inflation which allows us to compare costs of each event across time.
\subsubsection{Human Impact}
The data set provides three variables that are related to the direct impact to humans; fatalities, injured/effected, evacuated and utility. We composed a composite human impact score using these variables. We use equal weightings in the construction of the composite variable, however unequal weight could be given if a prior knowledge of a particular variable should be weighted more \cite{song2013composite}.
Since all of these variables are on vastly different scales of magnitude, we normalize the variables and add them to create our composite human impact score. 

\subsection{Predictors}
The predictors in this analysis are event type, earthquakes magnitude, and events duration.
Table \ref{cat} and \ref{numer} summarize the predictors.
\begin{itemize}
	\item \textbf{Event Type}: We include the event type as the main predictor as we predict that different types of events will result with differing effects on economic and human costs. For example, a biological pandemic will create more human loss than direct economic costs, whereas a flood will cause more economic damage than human loss.
	\item \textbf{Earthquakes Magnitude}: We assume that stronger earthquakes will cause greater economic and human loss.
	\item \textbf{Events Duration}: We predict that as an events duration increases, both measures will increase. 
\end{itemize}
\newpage
% latex table generated in R 3.4.4 by xtable 1.8-4 package
% Tue Mar 10 19:42:42 2020
\begingroup\footnotesize
\begin{longtable}{ll|rrr}
 \textbf{Variable} & \textbf{Levels} & $\mathbf{n}$ & $\mathbf{\%}$ & $\mathbf{\sum \%}$ \\ 
  \hline
Title Label & future & 2197 & 86.2 & 86.2 \\ 
   & life & 85 & 3.3 & 89.5 \\ 
   & new & 73 & 2.9 & 92.3 \\ 
   & world & 195 & 7.7 & 100.0 \\ 
   \hline
 & all & 2550 & 100.0 &  \\ 
   \hline
\hline
Video Theme & brain & 147 & 5.8 & 5.8 \\ 
   & business & 184 & 7.2 & 13.0 \\ 
   & culture & 605 & 23.7 & 36.7 \\ 
   & design & 329 & 12.9 & 49.6 \\ 
   & energy & 64 & 2.5 & 52.1 \\ 
   & global & 354 & 13.9 & 66.0 \\ 
   & health & 192 & 7.5 & 73.5 \\ 
   & music & 118 & 4.6 & 78.2 \\ 
   & science & 346 & 13.6 & 91.7 \\ 
   & social & 211 & 8.3 & 100.0 \\ 
   \hline
 & all & 2550 & 100.0 &  \\ 
   \hline
\hline
Video Age Label & new & 1711 & 67.1 & 67.1 \\ 
   & old & 839 & 32.9 & 100.0 \\ 
   \hline
 & all & 2550 & 100.0 &  \\ 
   \hline
\hline
\hline
\caption{} 
\label{cat}
\end{longtable}
\endgroup

% latex table generated in R 3.4.4 by xtable 1.8-4 package
% Mon Mar  9 18:17:55 2020
\begingroup\footnotesize
\begin{longtable}{lrrrrrrrrrr}
 \textbf{Variable} & $\mathbf{n}$ & \textbf{Min} & $\mathbf{q_1}$ & $\mathbf{\widetilde{x}}$ & $\mathbf{\bar{x}}$ & $\mathbf{q_3}$ & \textbf{Max} & $\mathbf{s}$ & \textbf{IQR} & \textbf{\#NA} \\ 
  \hline
Average Views/Day & 2550 &  17.0 &  311.0 &  724.0 & 1486.1 & 1752.8 & 28347.0 & 2148.2 & 1441.8 & 0 \\ 
  Video Duration & 2550 & 135.0 &  577.0 &  848.0 &  826.5 & 1046.8 &  5256.0 &  374.0 &  469.8 & 0 \\ 
  Num. Speakers & 2550 &   1.0 &    1.0 &    1.0 &    1.0 &    1.0 &     5.0 &    0.2 &    0.0 & 0 \\ 
  Film Age & 2550 & 126.0 & 1177.2 & 2100.0 & 2230.9 & 2977.0 & 16667.0 & 1385.9 & 1799.8 & 0 \\ 
  Title Length & 2550 &   1.0 &    5.0 &    6.0 &    6.2 &    8.0 &    16.0 &    2.3 &    3.0 & 0 \\ 
  Popularity & 2550 &   0.6 &    1.6 &    1.8 &    1.7 &    1.9 &     3.2 &    0.2 &    0.2 & 0 \\ 
  \hline
\caption{Numerical Response and Predictors} 
\label{numer}
\end{longtable}
\endgroup
















